\documentclass[11pt]{article}


\setlength{\parindent}{0pt}
\usepackage{xltxtra}
\usepackage{hyperref}
\hypersetup{hidelinks}
\usepackage{url}
\urlstyle{tt}
\usepackage{xcolor}
\definecolor{CVBlue}{RGB}{23,110,191}
\usepackage{calc}
\usepackage{graphicx}
\usepackage{tikz}
\usetikzlibrary{calc}
\usepackage{fontspec}
\usepackage{xeCJK}
\usepackage{enumitem}
\CJKsetecglue{} %% 取消中文与数字之间的间隙


%% 主文档字体设置
\setmainfont[
    Path = fonts/Main/,
    Extension = .otf,
    BoldFont = texgyretermes-bold.otf, % 加粗字体
]{texgyretermes-regular.otf} % 正文字体

% 中文字体设置
\setCJKmainfont[
    Path = fonts/hansans/,
    Extension = .ttf,
    BoldFont = NotoSansSC-Bold.ttf, % 加粗字体
]{NotoSansSC-Regular.otf} % 正文字体


\usepackage{relsize}
\usepackage{xspace}

% 使用 fontawesome(部分图标)
\usepackage{fontawesome} 

% A4纸,上下左右边距
\usepackage[
    a4paper,
    left=1.2cm,
    right=1.2cm,
    top=1.5cm,
    bottom=1cm,
    nohead
]{geometry}

\renewcommand{\baselinestretch}{1.5} % 行间距设为1.5

\usepackage{titlesec}
\usepackage{enumitem}
\setlist{noitemsep} % 取消列表项间的额外间距
%\setlist{nosep} % 取消所有垂直间距
\setlist[itemize]{topsep=0.25em, leftmargin=*}
\setlist[enumerate]{topsep=0.25em, leftmargin=*}

% --- 用于控制【不同项目之间】的垂直距离 ---
\newlength{\interProjectSpacing}
\setlength{\interProjectSpacing}{0.9em} % <--- 在此调整项目之间的距离
\newcommand{\projectsep}{\vspace{\interProjectSpacing}}

% --- 用于控制【项目标题】与下方【项目描述】的距离 ---
\newlength{\intraProjectTitleSep}
\setlength{\intraProjectTitleSep}{0.4em} % <--- 在此调整标题和描述的距离
\newcommand{\titlebreak}{\\[\intraProjectTitleSep]}

% --- 用于控制【项目描述】与下方【要点列表】的距离 ---
\newlength{\intraProjectListTopSep}
\setlength{\intraProjectListTopSep}{0.2em} % <--- 在此调整描述和列表的距离

% =======================================================================


\titleformat{\section}         % 定制 \section 命令 
{\large\bfseries\raggedright} % 将 section 标题设置为大号、粗体且左对齐
{}{0em}                      % 可用于为所有 section 添加前缀(如“章节...”)
{}                           % 可用于在标题前插入代码
[{\color{CVBlue}\titlerule}]  % 在标题后插入一条横线
\titlespacing*{\section}{0cm}{*1.6}{*1.2}



\begin{document}
\pagenumbering{gobble}

%%%% 利用tikz来定位照片
\begin{tikzpicture}[remember picture, overlay] 
    \node[anchor = north east] at ($(current page.north east)+(-2cm,-1.2cm)$) {\includegraphics[height=3cm]{avatar.jpg}};
  \end{tikzpicture}%
  %%%% 利用tikz来定位学校Logo,这里只在第一页显示
  \begin{tikzpicture}[remember picture, overlay] 
    \node[anchor = north west] at ($(current page.north west)+(0.5cm,+1.0cm)$) {\includegraphics[height=6cm]{zju.png}};
  \end{tikzpicture}%
\centerline{\LARGE\bfseries{吴秋丹}} 

\centerline{\normalsize{\faPhone\ 18157306768 \quad \faEnvelopeO\ \href{mailto:3230100423@zju.edu.cn}{3230100423@zju.edu.cn}}} 
    
\section{\makebox[\widthof{\faGraduationCap}][c]{\color{CVBlue}\faGraduationCap}\ 学业成绩}    
\textbf{浙江大学} \hfill 2023.9 -- 至今\\[0.5em] % 标题和正文间加一点距离
环境科学辅修大数据与数据科学\quad 大三 
\begin{itemize}[nosep]
    \item 均绩:4.73,每年专业排名均为第一
    \item 英语水平:CET-6:604,托福:108
    \item 核心课程:《物理化学》(5.0)、《有机化学》(5.0)、《环境物理学》(5.0)、《概率论与数理统计》(5.0)、《数据结构》(4.8)
\end{itemize}

\section{\makebox[\widthof{\faUsers}][c]{\color{CVBlue}\faUsers}\ 科研经历}

% --- 第一个项目 ---
% 将标题行末尾的 \\ 替换为 \titlebreak 命令
\textbf{校级srtp蒙特利尔议定书对全球微塑料分布的影响及建模——项目负责人} \hfill 2024.08 -- 至今 \titlebreak
项目描述:参与评估《蒙特利尔议定书》对全球微塑料污染间接影响的研究项目。负责收集并分析全球紫外线辐射数据,结合臭氧层变化构建\textbf{高分辨率网格化紫外线辐射数据集};整合不同紫外线强度下常见塑料(如聚乙烯、聚丙烯)的降解实验数据,采用数学建模方法建立\textbf{紫外线辐射—塑料降解—微塑料生成的过程模型};通过\textbf{情景分析}对比有无议定书两种情境下的未来微塑料污染趋势,量化该协议对全球微塑料污染的环境效益,为国际环境政策评估提供科学依据。
% 在 itemize 的选项中,使用 topsep=\intraProjectListTopSep 来控制上边距
\begin{itemize}[nosep, topsep=\intraProjectListTopSep]
    \item \textbf{核心产出}:掌握了从\textbf{数据采集、清洗、文献阅读、数学建模、编程}等科研方法与能力;撰写蒙特利尔议定书对微塑料污染的环境效益评估1份;学术论文一篇。

\end{itemize}

% 使用 \projectsep 命令来分隔两个项目
\projectsep

% --- 第二个项目 ---
\textbf{大模型的体验计算与设计课题组实习——界面信息一致性测评} \hfill 2025.11 -- 至今 \titlebreak
项目描述:参与大模型体验计算与设计课题组实习,专注于界面信息一致性测评任务。负责构建多模态测试数据集,设计一致性评估指标,并利用大模型对界面文案、标签及交互提示与设计规范进行自动化对比分析,识别信息不一致问题。通过定量评估与案例研究,总结出典型不一致类型,为模型迭代和设计优化提供数据支持,有效提升了界面信息的一致性与用户体验。
\begin{itemize}[nosep, topsep=\intraProjectListTopSep]
    \item \textbf{核心产出}:帮助构建了\textbf{多模态界面一致性测试数据集},涵盖文案、标签、交互提示等界面元素,为自动化测评提供基础;提升编程能力。
\end{itemize}

\section{\makebox[\widthof{\faGraduationCap}][c]{\color{CVBlue}\faList}\ 对外交流与获奖情况}
\begin{itemize}
    \item 新加坡南洋理工大学电子商务与社交媒体战略暑校 \hfill 2024.8
    \item 全国大学生数学竞赛初赛(非数学A类)三等奖 \hfill 2023.12
    \item 浙江省物理创新竞赛(理论)三等奖 \hfill 2023.12
    \item 美国大学生数学建模比赛 \hfill 2026.1
    \item 南都二等奖学金 \hfill 2024.10
    \item 校二等奖学金 \hfill 2024.10
    \item 省政府奖学金 \hfill 2025.10
    \item 环境科技奖学金 \hfill 2025.10
    
\end{itemize}
\end{document}
